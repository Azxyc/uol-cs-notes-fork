\documentclass{article}
\usepackage{float}

\title{COMP111 - Exercise 6 Answers}
\author{Ben Weston}
\date{\today}

\begin{document}
\maketitle
\begin{enumerate}
        \item
                \begin{enumerate}
                        \item This statement is false as not all of the statements are subsets of $p_3$:
                                \begin{table}[H]
                                        \centering
                                        \begin{tabular}{|c|c|c|c|c|}
                                                \hline
                                                $p_1$ & $p_2$ & $p_3$ & $(p_1\Rightarrow p_2)$ & $(p_2\Rightarrow p_3)$\\
                                                \hline\hline
                                                0 & 0 & 0 & 1 & 1\\
                                                \hline
                                                0 & 0 & 1 & 1 & 1\\
                                                \hline
                                                0 & 1 & 0 & 1 & 0\\
                                                \hline
                                                0 & 1 & 1 & 1 & 1\\
                                                \hline
                                                1 & 0 & 0 & 0 & 1\\
                                                \hline
                                                1 & 0 & 1 & 0 & 1\\
                                                \hline
                                                1 & 1 & 0 & 1 & 0\\
                                                \hline
                                                1 & 1 & 1 & 1 & 1\\
                                                \hline
                                        \end{tabular}
                                \end{table}
                \end{enumerate}
                \setcounter{enumi}{2}
        \item
                \begin{enumerate}\setcounter{enumii}{3}
                        \item $S=\{HHHH,HHHT,HHTH,HHTT,HTHH,HTHT,HTTH,\\HTTT,THHH,THHT,THTH,THTT,TTHH,TTHT,TTTH,\\TTTT\}$ where $P(x)=\frac{1}{16}$ for all $x\in S$.

                                $P(T\geq3|S')$ where $S'=\{HTHH,HTHT,HTTH,HTTT,THHH,\\THHT,THTH,THTT,TTHH,TTHT,TTTH,TTTT\}$

                                Therefore the probability is $\frac{\frac{5}{16}}{\frac{12}{16}}=\frac{5}{12}$
                        \item No, as there are only four coins, two of which are high. In order for three coins to come up tails then at least one must be a low coin. Therefore if you know that three coins are tails you can conclude that at least one value coin was chosen proving that they are not independent.
                \end{enumerate}
\end{enumerate}
\end{document}
