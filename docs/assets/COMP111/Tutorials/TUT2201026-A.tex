\documentclass{article}
\usepackage{caption,float}

\title{COMP111 - Tutorial 2 Answers}
\author{Ben Weston}
\date{\today}

\begin{document}
\maketitle
\begin{enumerate}\setcounter{enumi}{9}
	\item{Starting with a in the frontier all children are expanded. This removes a from the frontier and adds all the children paths to the frontier. As a only has one child, b, the path ab is added to the frontier. As this is a breadth first search the first node to have been put in the frontier is the next to be expanded. This continues in table \ref{q10}:

\begin{table}[H]
\centering
	\begin{tabular}{|l|l|}
	\hline
	Expanded Path & Frontiers\\
	\hline\hline
	& a\\
	\hline
	a, not goal state & ab\\
	\hline
	ab, not goal state & aba, abc\\
	\hline
	aba, not goal state & abc, abab\\
	\hline
	abc is goal & abab\\
	\hline
\end{tabular}
\caption{}
\label{q10}
\end{table}}

\item{The sequence of events is the same as with a breadth-first search however instead of taking the first node the last node is used. For successor states that are added in the same expansion, the order of their expansion doesn't matter. I will take advantage of this to create the two sequences in tables \ref{short} and \ref{long}:

\begin{table}[H]
\centering
\begin{tabular}{|l|l|}
	\hline
	Expanded Path & Frontiers\\
	\hline\hline
	& a\\
	\hline
	a, not goal state & ab\\
	\hline
	ab, not goal state & aba, abc\\
	\hline
	abc is goal & aba\\
	\hline
\end{tabular}
\caption{}\label{short}
\end{table}

\begin{table}[H]
\centering
\begin{tabular}{|l|l|}
	\hline
	Expanded Path & Frontiers\\
	\hline\hline
	& a\\
	\hline
	a, not goal state & ab\\
	\hline
	ab, not goal state & abc, aba\\
	\hline
	aba, not goal state & abc, abab\\
	\hline
	abab, not goal state & abc, ababa, ababc\\
	\hline
	ababc is goal & abc, ababa\\
	\hline
\end{tabular}
\caption{}\label{long}
\end{table}}
\end{enumerate}
\end{document}
