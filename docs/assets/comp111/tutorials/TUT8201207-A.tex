\documentclass{article}
\usepackage{amsmath}
\title{COMP111 - Exercise 8 Answers}
\author{Ben Weston}
\date{\today}

\begin{document}
\maketitle
\begin{enumerate}
        \item
                \begin{enumerate}
                        \item{
                                        The joint probability distribution of $\mathbf{P}($Country, Mothertongue$)$ is represented by the following table where Country = C and Mothertongue = Mt:
                                                
                                        \begin{center}
                                                \begin{tabular}{|l|c|c|c|}
                                                        \hline
                                                        & C = Eng & C = Scot & C = Welsh\\
                                                        \hline
                                                        Mt = Eng & 0.836 & 0.056 & 0.024\\
                                                        \hline
                                                        Mt = Scot & 0.0352 & 0.024 & 0.0\\
                                                        \hline
                                                        Mt = Welsh & 0.0088 & 0 & 0.016\\
                                                        \hline
                                                \end{tabular}
                                        \end{center}
                                }
                        \item You can sum the rows from the table to give: 
                                $$\mathbf{P}(\text{Mothertongue}) = (0.916,0.0592,0.0248)$$
                                where Mothertongue takes the values Eng, Scot and Welsh.
                \end{enumerate}
\end{enumerate}
\end{document}
